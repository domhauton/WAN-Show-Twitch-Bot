\documentclass[10pt]{article}
\author{Ghost}
\title{LTT Twitch Chat Bot Manual}
\begin{document}
\maketitle
\tableofcontents
\section{Host Commands}
Hosts (Tier0) have no special commands. Links posted by hosts will can be used by the !lll command and if a host's post contains "http://" or "https://" the link will be repeated by the bot. 
\section{Bot Operator Commands}
Commands for Tier1 users and above.
\subsection{Blacklisting Words}
To black list a word use the syntax \emph{"!bot blw word"} where "word" would be blacklisted. The bot will inform you if the word is already on the blacklist and if it has been added successfully.
\subsection{Removing BlackListed Words}
To remove a blacklisted word use the command \emph{"!bot rmblw word"} where "word" would be removed from the blacklist. The bot will inform you if the word was actually on the blacklist or if the the word was not found on the blacklist.
\subsection{Making users an Operator}
There are 3 tiers of operators.
\begin{itemize}
\item {Tier0 - Used for the channel hosts}
\item {Tier1 - Used for bot operators}
\item {Tier2 - Gives users bot functionality}
\item {Tier3 - Used for IRC chat moderators, gives them immunity from bot bans}
\end{itemize}
To make someone an operator use the command \emph{"!bot addop tier\# username"}. The tier must be 0 - 3 and the username must be lowercase.
\subsection{Removing Operators}
To remove an operator use the command \emph{"!bot rm username"}. This command will not allow you to remove the final tier0 operator.
\subsection{Set the show start time}
To set the show start time for the timestamp functionality use the command \emph{"!bot sstart"}.
\subsection{Setting bot variables}
To set a bot variable use the command \emph{"!bot set variablename value"}.\\
\begin{tabular}{| l | l |}
\hline
Variable Name & Meaning\\ \hline
maxMsg & The number of messages the program will store per user.\\ \hline
linkRepeatCountHost & Number of times host links will be repeated\\ \hline
linkRepeatCountMod & Number of times operator links will be repeated.\\ \hline
voteBanMax & Number of !reports required for you to be notified.\\ \hline
rPostVal & Number of messages used to average posting speed.\\ \hline
secpermsg & The minimum seconds between messages allowed.\\ \hline
longestSubStringAllowed & Longest common substring allowed in messages.\\ \hline
repetitionSearch & How many messages are checked for common substrings.\\ \hline
\end{tabular}
\section{Bot Moderator Commands}
Commands for Tier2 users and above
\subsection{Repeating Links}
Tier0 users only need to post a link (a message containing "http://" or "https://"\\
Other users need to use the command \emph{"!link url"}.
\section{User Commands}
Commands for any user.
\subsection{Last Linus Link}
Command: !lll\\
Replies with the message: "[BOT] Linus' Last Link: http://example.com"
The command will let you know if the there is no last link stored.
\subsection{Time Till Live}
Command: !ttl\\
Replies with the message: "[BOT] The next WAN Show should begin in: \# day \# hours and \# minutes."
The command is deactivated for the 24 hours after the start of the WAN show.
\subsection{Help Command}
Command: !help\\
Displays: "[BOT] You can find out more about the bot here: http://goo.gl/jZ1CmS. If you want to request an unban please tweet @deadfire19"
\subsection{Report}
Command: !report username\\
If you report a user the bot operators will be notified.
\subsection{Timestamps}
The bot allows you to generate timestamps during the show to help with the YouTube description. Each submitters timestamps are kept separate and everyone's timestamps are credited. Linus will most likely put the best set of timestamps in the description. At the start of the show the show start time will be set. You will see a print out from the bot: "[BOT] Show Start time has been set." From that point on your timestamps will be recorded.\\\\
 
To record a time stamp use the command \emph{"!ts topicname"} for example posting  "!ts Apple goes Bankrupt" would store the time stamp "00:00:00 Apple goes Bankrupt" under your name.
 \\\\
When you have finished recording timestamps use the command \emph{"!ts save"}. This will dump save your timestamps into the file linus will take timestamps from. Only do this once at the end. This will give you the printout "[BOT] You have submitted \# timestamps." If you do not do this you timestamps will not be saved!
\end{document}